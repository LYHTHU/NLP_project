%
% File acl2015.tex
%
% Contact: car@ir.hit.edu.cn, gdzhou@suda.edu.cn
%%
%% Based on the style files for ACL-2014, which were, in turn,
%% Based on the style files for ACL-2013, which were, in turn,
%% Based on the style files for ACL-2012, which were, in turn,
%% based on the style files for ACL-2011, which were, in turn, 
%% based on the style files for ACL-2010, which were, in turn, 
%% based on the style files for ACL-IJCNLP-2009, which were, in turn,
%% based on the style files for EACL-2009 and IJCNLP-2008...

%% Based on the style files for EACL 2006 by 
%%e.agirre@ehu.es or Sergi.Balari@uab.es
%% and that of ACL 08 by Joakim Nivre and Noah Smith

\documentclass[11pt]{article}
\usepackage{acl2015}
\usepackage{times}
\usepackage{url}
\usepackage{latexsym}
% \usepackage{cite}

%\setlength\titlebox{5cm}

% You can expand the titlebox if you need extra space
% to show all the authors. Please do not make the titlebox
% smaller than 5cm (the original size); we will check this
% in the camera-ready version and ask you to change it back.


\title{Text mining on Novels}

\author{Xiangyu Wei \\
  CIMS \\
  New York University \\
  {\tt xw1882@nyu.edu} \\\And
  Yunhao Li \\ 
  CIMS \\
  New York University \\
  {\tt yl6220@nyu.edu} \\\And
  Kaili Ding \\ 
  CIMS \\
  New York University \\
  {\tt kd2164@nyu.edu} \\}

\date{}

\begin{document}
\maketitle
\begin{abstract}
  We develop a system that could automatically mining the data of novels. The system includes name entity recognition and merging, coreference resolution, interaction and sentiment analysis and relation extraction. Finally, we test it on two novels and the result shows that our system runs well.
\end{abstract}

\section{Introduction}

The information extraction is an interesting field in Natural Language Processing(NLP). And text mining is an information extraction process for deriving high-quality information from text. As one of the most popular genres of literature, novel always have a lot of information between roles. Remembering all the roles and their relationships is somewhat hard for readers. A system that can help the readers understand the novel will be great. \\
In this paper we aims at mining the sentiment and social relationships between roles of the novels. 

\section{Related work}

The following instructions are directed to authors of papers submitted
to ACL-2015 or accepted for publication in its proceedings. All
authors are required to adhere to these specifications. Authors are
required to provide a Portable Document Format (PDF) version of their
papers. \textbf{The proceedings are designed for printing on A4
paper.}

We will make more detailed instructions available at
\url{http://acl2015.org/publication.html}. Please check this website 
regularly.


\section{Approaches}

It is a pipeline that works on processing the novel in our work. It can be split into several parts introduced in the following sections.

\subsection{Role entity recognition and merging} % Kaili Ding

We strongly prefer that you prepare your PDF files using \LaTeX\ with
the official ACL 2015 style file (acl2015.sty) and bibliography style
(acl.bst). These files are available at
\url{http://acl2015.org}. You will also find the document
you are currently reading (acl2015.pdf) and its \LaTeX\ source code
(acl2015.tex) on this website.

You can alternatively use Microsoft Word to produce your PDF file. In
this case, we strongly recommend the use of the Word template file
(acl2015.dot) on the ACL 2015 website (\url{http://acl2015.org}). 
If you have an option, we recommend that you use the \LaTeX2e version. 
If you will be using the Microsoft Word template, we suggest that you 
anonymize your source file so that the pdf produced does not retain your
identity.  This can be done by removing any personal information
from your source document properties.



\subsection{Coreference resolution} % Xiangyu Wei
\label{sect:pdf}

For the production of the electronic manuscript you must use Adobe's
Portable Document Format (PDF). PDF files are usually produced from
\LaTeX\ using the \textit{pdflatex} command. If your version of
\LaTeX\ produces Postscript files, you can convert these into PDF
using \textit{ps2pdf} or \textit{dvipdf}. On Windows, you can also use
Adobe Distiller to generate PDF.

Please make sure that your PDF file includes all the necessary fonts
(especially tree diagrams, symbols, and fonts with Asian
characters). When you print or create the PDF file, there is usually
an option in your printer setup to include none, all or just
non-standard fonts.  Please make sure that you select the option of
including ALL the fonts. \textbf{Before sending it, test your PDF by
  printing it from a computer different from the one where it was
  created.} Moreover, some word processors may generate very large PDF
files, where each page is rendered as an image. Such images may
reproduce poorly. In this case, try alternative ways to obtain the
PDF. One way on some systems is to install a driver for a postscript
printer, send your document to the printer specifying ``Output to a
file'', then convert the file to PDF.

It is of utmost importance to specify the \textbf{A4 format} (21 cm
x 29.7 cm) when formatting the paper. When working with
{\tt dvips}, for instance, one should specify {\tt -t a4}.
Or using the command \verb|\special{papersize=210mm,297mm}| in the latex
preamble (directly below the \verb|\usepackage| commands). Then using 
{\tt dvipdf} and/or {\tt pdflatex} which would make it easier for some.


Print-outs of the PDF file on A4 paper should be identical to the
hardcopy version. If you cannot meet the above requirements about the
production of your electronic submission, please contact the
publication chairs as soon as possible.


\subsection{Text processing} % Yunhao Li
\label{ssec:layout}

After the previous processing, we now have a text where all the occurring pronouns replaced by a most likely name, and a name dictionary that maps all the alias of the same role to a single formatted name. Now we replaced all the names with its formatted name. Considering the situation that one name could be a sub-string of other names or words, we use a very simple but effective trick to solve this problem. \\
First, we create a dictionary, where the keys are the alias(i.e, names occur in the novel) and the correlated values are the formatted names. Then we sorted the item of the dictionary by the length of key in descending order. Then for each line in the text, we iterate the dictionary in order and replace the word that matches the key with its value. In particular, we avoid replacing the sequences occurring as part of a word but also matching a key(e.g, "Ted", may match "created", "Don" may match "Don't", etc). We make a simple rule: we only replace the matching sequences that are not followed by an letter and not start with a double quote('"'), which works very well. \\ 

Then we are going to split the text into sentences. We use NLTK and coreNLP to do this work. The list of sentences will be useful in the following sentiment analysis and relation extraction part.

\subsection{The analysis of interaction and sentiment between roles} % Yunhao Li
First we need to find the sentences of interests. The program will iterate the sentences list, and gather the sentences where two or more roles show up.

\subsection{Relation extraction} % Xiangyu Wei
\label{ssec:first}

Center the title, author's name(s) and affiliation(s) across both
columns. Do not use footnotes for affiliations. Do not include the
paper ID number assigned during the submission process. Use the
two-column format only when you begin the abstract.

{\bf Title}: Place the title centered at the top of the first page, in
a 15-point bold font. (For a complete guide to font sizes and styles,
see Table~\ref{font-table}) Long titles should be typed on two lines
without a blank line intervening. Approximately, put the title at 2.5
cm from the top of the page, followed by a blank line, then the
author's names(s), and the affiliation on the following line. Do not
use only initials for given names (middle initials are allowed). Do
not format surnames in all capitals (e.g., use ``Schlangen'' not
``SCHLANGEN'').  Do not format title and section headings in all
capitals as well except for proper names (such as ``BLEU'') that are
conventionally in all capitals.  The affiliation should contain the
author's complete address, and if possible, an electronic mail
address. Start the body of the first page 7.5 cm from the top of the
page.

The title, author names and addresses should be completely identical
to those entered to the electronical paper submission website in order
to maintain the consistency of author information among all
publications of the conference. If they are different, the publication
chairs may resolve the difference without consulting with you; so it
is in your own interest to double-check that the information is
consistent.

{\bf Abstract}: Type the abstract at the beginning of the first
column. The width of the abstract text should be smaller than the
width of the columns for the text in the body of the paper by about
0.6 cm on each side. Center the word {\bf Abstract} in a 12 point bold
font above the body of the abstract. The abstract should be a concise
summary of the general thesis and conclusions of the paper. It should
be no longer than 200 words. The abstract text should be in 10 point font.

{\bf Text}: Begin typing the main body of the text immediately after
the abstract, observing the two-column format as shown in 
the present document. Do not include page numbers.

{\bf Indent} when starting a new paragraph. Use 11 points for text and 
subsection headings, 12 points for section headings and 15 points for
the title. 

\subsection{Sections}

{\bf Headings}: Type and label section and subsection headings in the
style shown on the present document.  Use numbered sections (Arabic
numerals) in order to facilitate cross references. Number subsections
with the section number and the subsection number separated by a dot,
in Arabic numerals. Do not number subsubsections.

{\bf Citations}: Citations within the text appear in parentheses
as~\cite{Gusfield:97} or, if the author's name appears in the text
itself, as Gusfield~\shortcite{Gusfield:97}.  Append lowercase letters
to the year in cases of ambiguity.  Treat double authors as
in~\cite{Aho:72}, but write as in~\cite{Chandra:81} when more than two
authors are involved. Collapse multiple citations as
in~\cite{Gusfield:97,Aho:72}. Also refrain from using full citations
as sentence constituents. We suggest that instead of
\begin{quote}
  ``\cite{Gusfield:97} showed that ...''
\end{quote}
you use
\begin{quote}
``Gusfield \shortcite{Gusfield:97}   showed that ...''
\end{quote}

If you are using the provided \LaTeX{} and Bib\TeX{} style files, you
can use the command \verb|\newcite| to get ``author (year)'' citations.

As reviewing will be double-blind, the submitted version of the papers
should not include the authors' names and affiliations. Furthermore,
self-references that reveal the author's identity, e.g.,
\begin{quote}
``We previously showed \cite{Gusfield:97} ...''  
\end{quote}
should be avoided. Instead, use citations such as 
\begin{quote}
``Gusfield \shortcite{Gusfield:97}
previously showed ... ''
\end{quote}

\textbf{Please do not use anonymous citations} and do not include
acknowledgements when submitting your papers. Papers that do not
conform to these requirements may be rejected without review.

\textbf{References}: Gather the full set of references together under
the heading {\bf References}; place the section before any Appendices,
unless they contain references. Arrange the references alphabetically
by first author, rather than by order of occurrence in the text.
Provide as complete a citation as possible, using a consistent format,
such as the one for {\em Computational Linguistics\/} or the one in the 
{\em Publication Manual of the American 
Psychological Association\/}~\cite{APA:83}.  Use of full names for
authors rather than initials is preferred.  A list of abbreviations
for common computer science journals can be found in the ACM 
{\em Computing Reviews\/}~\cite{ACM:83}.

The \LaTeX{} and Bib\TeX{} style files provided roughly fit the
American Psychological Association format, allowing regular citations, 
short citations and multiple citations as described above.

{\bf Appendices}: Appendices, if any, directly follow the text and the
references (but see above).  Letter them in sequence and provide an
informative title: {\bf Appendix A. Title of Appendix}.

\section{Experiment}

Following ACL 2014 we will also we will attempt to automatically convert 
your \LaTeX\ source files to publish papers in machine-readable 
XML with semantic markup in the ACL Anthology, in addition to the 
traditional PDF format.  This will allow us to create, over the next 
few years, a growing corpus of scientific text for our own future research, 
and picks up on recent initiatives on converting ACL papers from earlier 
years to XML. 

We encourage you to submit a ZIP file of your \LaTeX\ sources along
with the camera-ready version of your paper. We will then convert them
to XML automatically, using the LaTeXML tool
(\url{http://dlmf.nist.gov/LaTeXML}). LaTeXML has \emph{bindings} for
a number of \LaTeX\ packages, including the ACL 2015 stylefile. These
bindings allow LaTeXML to render the commands from these packages
correctly in XML. For best results, we encourage you to use the
packages that are officially supported by LaTeXML, listed at
\url{http://dlmf.nist.gov/LaTeXML/manual/included.bindings}

\begin{table}[h]
\begin{center}
\begin{tabular}{|l|rl|}
\hline \bf Type of Text & \bf Font Size & \bf Style \\ \hline
paper title & 15 pt & bold \\
author names & 12 pt & bold \\
author affiliation & 12 pt & \\
the word ``Abstract'' & 12 pt & bold \\
section titles & 12 pt & bold \\
document text & 11 pt  &\\
captions & 11 pt & \\
abstract text & 10 pt & \\
bibliography & 10 pt & \\
footnotes & 9 pt & \\
\hline
\end{tabular}
\end{center}
\caption{\label{font-table} Font guide. }
\end{table}





\section{Discussions} % Yunhao Li


\section{Conclusions}
\label{sec:length}
We develop a system that could automatically mining the data of novels, including name entity recognition and merging, coreference resolution, interaction and sentiment analysis and relation extraction. And we test it on two novels, \textit{Harry Potter and the Sorcerer's Stone} and \textit{prejudice and discrimination}. The result shows that our system runs well.

\section*{Acknowledgments} % Yunhao Li

We really appreciate the work of Stanford NLP group. Their CoreNLP and stanfordnlp library really help a lot. We are also grateful to the researchers and developers of Spacy and NLTK. Without their great job we cannot finish our work.

% include your own bib file like this:
%\bibliographystyle{acl}
%\bibliography{acl2015}

\begin{thebibliography}{}

\bibitem[\protect\citename{Aho and Ullman}1972]{Aho:72}
Alfred~V. Aho and Jeffrey~D. Ullman.
\newblock 1972.
\newblock {\em The Theory of Parsing, Translation and Compiling}, volume~1.
\newblock Prentice-{Hall}, Englewood Cliffs, NJ.

\bibitem[\protect\citename{{American Psychological Association}}1983]{APA:83}
{American Psychological Association}.
\newblock 1983.
\newblock {\em Publications Manual}.
\newblock American Psychological Association, Washington, DC.

\bibitem[\protect\citename{{Association for Computing Machinery}}1983]{ACM:83}
{Association for Computing Machinery}.
\newblock 1983.
\newblock {\em Computing Reviews}, 24(11):503--512.

\bibitem[\protect\citename{Chandra \bgroup et al.\egroup }1981]{Chandra:81}
Ashok~K. Chandra, Dexter~C. Kozen, and Larry~J. Stockmeyer.
\newblock 1981.
\newblock Alternation.
\newblock {\em Journal of the Association for Computing Machinery},
  28(1):114--133.

\bibitem[\protect\citename{Gusfield}1997]{Gusfield:97}
Dan Gusfield.
\newblock 1997.
\newblock {\em Algorithms on Strings, Trees and Sequences}.
\newblock Cambridge University Press, Cambridge, UK.
\end{thebibliography}

\end{document}